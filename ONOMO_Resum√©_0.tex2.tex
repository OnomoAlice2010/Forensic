\documentclass[12pt,a4paper]{article}

% --- Packages généraux ---
\usepackage[utf8]{inputenc}
\usepackage[T1]{fontenc}
\usepackage[french]{babel}
\usepackage[a4paper, top=1in, bottom=1in, left=1in, right=1in]{geometry}
\usepackage{array}
\usepackage{graphicx}
\usepackage{setspace}
\usepackage{titlesec}
\usepackage{amsmath}

% --- Paramètres de mise en page ---
\sloppy
\setlength{\emergencystretch}{3em}
\setstretch{1.5}

\begin{document}

% ============================
% PAGE DE GARDE
% ============================
\thispagestyle{empty} % Pas de numéro de page

% --- En-tête bilingue avec logo ---
\begin{center}
  \begin{minipage}{0.35\linewidth}
    \centering
    \textbf{RÉPUBLIQUE DU CAMEROUN} \\ 
    Paix - Travail - Patrie \\[1em]
    \textbf{------------} \\[1em]
    Université de Yaoundé I \\[1em]
    \textbf{------------} \\[1em]
    École Nationale Supérieure Polytechnique \\[1em]
    De Yaoundé \\[1em]
    \textbf{------------} \\
  \end{minipage}
  \hfill
  \begin{minipage}{0.25\linewidth}
    \centering
    \includegraphics[width=3cm]{logo.jpg}
  \end{minipage}
  \hfill
  \begin{minipage}{0.35\linewidth}
    \centering
    \textbf{REPUBLIC OF CAMEROON} \\ 
    Peace - Work - Fatherland \\[1em]
    \textbf{------------} \\[1em]
    The University of Yaoundé I \\[1em]
    \textbf{------------} \\[1em]
    National Advanced School of \\[1em]
    Engineering of Yaoundé \\[1em]
    \textbf{------------} \\
  \end{minipage}
\end{center}

\vspace{2.5cm}
% --- Titre principal ---
\begin{center}
  {\Huge \textbf{THÉORIE ET PRATIQUE DE \\[0.5em] L’INVESTIGATION NUMÉRIQUE}}
\end{center}
\vspace{2cm}

% --- Réalisé par ---
\noindent
\textbf{RÉALISÉ PAR :} \\[1em]
\renewcommand{\arraystretch}{1.3}
\begin{tabular}{|c|m{6cm}|c|c|}
  \hline
  \textbf{N°} & \textbf{NOMS ET PRÉNOMS} & \textbf{SPÉCIALITÉ} & \textbf{MATRICULE} \\
  \hline
  01 & ONOMO NGONO ALICE  & CIN-4 & 24P830 \\
  \hline
\end{tabular}

\vspace{1.5cm}

\noindent
\textbf{Examinateur :} \quad Mr MINKA Thierry 

\vfill % pousse le texte en bas de page

\begin{center}
  \Large \textbf{Année Académique: 2025/2026}
\end{center}

\newpage % ---- Saut de page vers le rapport ----

% ============================
% RAPPORT / RÉSUMÉ
% ============================

\begin{center}
\Huge Résumé de \og Théories et Pratiques de l’Investigation Numérique \fg
\end{center}

\vspace{1cm}

\section*{Avant-propos}
Cet ouvrage synthétise vingt années de recherche et de pratique en cybersécurité et en investigation numérique. 
Il s’adresse aux ingénieurs souhaitant se spécialiser dans l’investigation post-quantique, tout en abordant 
les enjeux juridiques liés à l’opposabilité des preuves numériques. 
Une contribution majeure réside dans la formalisation du \textbf{trilemme CRO} : 
\textit{Confidentialité, Fiabilité et Opposabilité juridique}.

\section*{I. Engagements déontologiques}
L’investigation numérique requiert une éthique stricte. 
Chaque outil ou méthode confère un pouvoir particulier sur les systèmes analysés. 
Un contrat déontologique engage ainsi le praticien à respecter la confidentialité, 
préserver l’intégrité et agir avec responsabilité.

\section{La société numérique : nouveau terrain de l’être}
\subsection{Transformation de l’existence}
Le numérique transforme l’ontologie humaine. L’individu existe à la fois physiquement et 
numériquement. Trois concepts s’imposent :
\begin{itemize}
\item \textbf{Ontologie numérique} : l’identité numérique comme prolongement de l’existence.
\item \textbf{Phénoménologie des données} : la trace comme matérialisation de la présence.
\item \textbf{Métaphysique digitale} : de nouvelles formes de relations et d’être.
\end{itemize}

\subsection{Le paradoxe de la transparence}
La quête de transparence entre en tension avec le droit à l’intimité. 
L’investigateur numérique agit comme médiateur entre vérité et vie privée.

\subsection{Épistémologie de la preuve numérique}
La transition de la preuve matérielle à la preuve numérique bouleverse les critères 
traditionnels de fiabilité et de recevabilité.

\section{II. Fondements et évolution de l’investigation numérique}
L’investigation repose sur une philosophie de la vérité adaptée à l’ère digitale. 
Les mathématiques (théorie de l’information, graphes, chaos) structurent les analyses, 
tandis que la révolution quantique transforme les paradigmes.

\subsection{Crise de la vérité numérique}
Trois défis émergent :
\begin{itemize}
\item \textbf{Manipulation algorithmique} : deepfakes et falsifications.
\item \textbf{Érosion de l’autorité} : pluralité des sources et confusion.
\item \textbf{Fragmentation du réel} : coexistence de réalités multiples.
\end{itemize}

\subsection{Fondements mathématiques}
\begin{itemize}
\item \textbf{Théorie de l’information} : entropie de Shannon
\[ H(X) = -\sum_{i=1}^{n} P(x_i) \log_2 P(x_i) \]
\item \textbf{Théorie des graphes} : représentation relationnelle
\[ G = (V, E) \]
\item \textbf{Théorie du chaos} : sensibilité aux conditions initiales
\[ \delta(t) \approx \delta(0)e^{\lambda t} \]
\end{itemize}

\subsection{Révolution quantique}
La mécanique quantique oppose déterminisme et probabilisme, localité et non-localité, 
certitude et incertitude. Elle introduit des notions comme la superposition et 
l’observateur participatif, redéfinissant la preuve numérique.

\subsection{Paradoxe de l’authenticité invisible}
Une tension existe entre authenticité, confidentialité et opposabilité. 
Plus une preuve est fiable, plus elle révèle d’informations, menaçant la confidentialité. 
Cette relation est exprimée par :
\[
\Delta A \cdot \Delta C \geq \hbar_{num}
\]
Les protocoles \textbf{Zero-Knowledge Non-Repudiation (ZK-NR)} permettent une résolution 
en vérifiant sans divulguer.

\section{III. Éthique et responsabilité}
L’investigateur devient un \textit{philosophe-praticien}. 
Il navigue entre trois tensions éthiques :
\begin{itemize}
\item Transparence vs vie privée.
\item Efficacité vs proportionnalité.
\item Innovation vs responsabilité.
\end{itemize}
La trace numérique est ainsi considérée comme manifestation d’existence.

\subsection{Vers une éthique post-quantique}
De nouveaux impératifs émergent : produire des preuves résistantes aux attaques quantiques 
et pratiquer une \textit{praxis de liberté}, garantissant mémoire collective et vérité.

\section*{IV. Cadre théorique et normes}
Les standards internationaux (ISO/IEC, méthodologies SANS et CERT) guident la collecte 
et l’analyse. L’accent est mis sur l’intégration d’éthique et de légalité 
dans les processus techniques.

\section*{V. Applications pratiques}
Des études de cas illustrent les apports concrets : cybercriminalité, cyberespionnage, 
fraudes financières. Chaque exemple montre la pertinence des concepts théoriques 
et la complexité de leur mise en œuvre.

\section*{Conclusion et perspectives}
Le manuel insiste sur l’importance d’une approche intégrée, technique et éthique. 
Les défis de l’informatique quantique imposent une adaptation constante des méthodes 
et une coopération internationale renforcée. 
L’investigation numérique devient ainsi une discipline au service de la vérité et de la justice.

\end{document}
