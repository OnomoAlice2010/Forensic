\documentclass[12pt,a4paper]{article}

\usepackage[utf8]{inputenc}   % Encodage
\usepackage[T1]{fontenc}      % Polices
\usepackage[french]{babel}    % Langue
\usepackage{geometry}         % Marges
\usepackage{array}            % Tableaux
\usepackage{graphicx}         % Images

\geometry{margin=1in}

\begin{document}
\thispagestyle{empty} % Pas de numéro de page

% --- En-tête bilingue avec logo ---
\begin{center}
  \begin{minipage}{0.35\linewidth}
    \centering
    \textbf{RÉPUBLIQUE DU CAMEROUN} \\ 
    Paix - Travail - Patrie \\[1em]
    \textbf{------------} \\[1em]
    Université de Yaoundé I \\[1em]
    \textbf{------------} \\[1em]
    École Nationale Supérieure Polytechnique \\[1em]
    De Yaoundé \\[1em]
    \textbf{------------} \\
  \end{minipage}
  \hfill
  \begin{minipage}{0.25\linewidth}
    \centering
    % Mets ton logo ENSPY dans le même dossier que ton fichier .tex
    % et appelle-le logo.png (ou adapte le nom ici)
    \includegraphics[width=3cm]{LOGO-POLYTECHNIQUE-01-scaled.jpg}
  \end{minipage}
  \hfill
  \begin{minipage}{0.35\linewidth}
    \centering
    \textbf{REPUBLIC OF CAMEROON} \\ 
    Peace - Work - Fatherland \\[1em]
    \textbf{------------} \\[1em]
    The University of Yaoundé I \\[1em]
    \textbf{------------} \\[1em]
    National Advanced School of \\[1em]
    Engineering of Yaoundé \\[1em]
    \textbf{------------} \\
  \end{minipage}
\end{center}

\vspace{3cm}

% --- Titre principal ---
\begin{center}
  {\Huge \textbf{THÉORIE ET PRATIQUE DE \\[0.5em] L’INVESTIGATION NUMÉRIQUE}}
\end{center}

\vspace{3cm}

% --- Réalisé par ---
\noindent
\textbf{RÉALISÉ PAR :} \\[1em]
\renewcommand{\arraystretch}{1.3}
\begin{tabular}{|c|m{6cm}|c|c|}
  \hline
  \textbf{N°} & \textbf{NOMS ET PRÉNOMS} & \textbf{SPÉCIALITÉ} & \textbf{MATRICULE} \\
  \hline
  01 & ONOMO NGONO ALICE  & CIN-4 & 24P830 \\
  \hline
\end{tabular}

\vspace{2cm}

\noindent
\textbf{Examinateur :} \quad Mr MINKA Thierry 

\vfill % pousse vers le bas

\begin{center}
  \Large \textbf{Année Académique: 2025/2026}
\end{center}

\end{document}
