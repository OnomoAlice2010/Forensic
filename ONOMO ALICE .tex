

\documentclass[12pt,a4paper]{article}
\usepackage[utf8]{inputenc}
\usepackage[T1]{fontenc}
\usepackage[french]{babel}
\usepackage{graphicx}
\usepackage{xcolor}
\usepackage{geometry}
\geometry{margin=2.5cm}
\usepackage{amsmath}
\usepackage{textcomp}
\usepackage{hyperref}
\usepackage{float}

\title{Partie 1 : Fondements philosophiques et épistémologiques}
\author{Votre Nom} % Remplacer par votre nom
\date{\today}

\begin{document}
\maketitle

\section{Analyse Critique du Paradoxe de la Transparence}

Le paradoxe de la transparence, identifié par Byung-Chul Han, démontre les tensions qui existent entre la quête de transparence numérique et le droit à l'intimité. Cependant, dans nos cultures modernes, où le numérique joue un rôle essentiel, cette poursuite de notoriété peut se montrer parfois éphémère et problématique.
De prime abord, la transparence présentée comme une valeur positive, est devenue un impératif dans plusieurs domaines, allant de la gouvernance à la vie quotidienne. Ainsi, dans le gouvernement, par exemple, une transparence renforcée est souvent perçue comme égale de la responsabilité et de la démocratie. Par ailleurs, ce besoin de clarté peut mener à un contrôle pouvant nuire à la vie privée des citoyens. 
En définitive, il était question de faire une analyse sur la transparence identifiée par Byung-Chul Han.Néamoins, nous nous posons la question de savoir dans quelles conditions cette transparence est-elle acquise ? Les données personnelles collectées par les gouvernements, sous prétexte de sécurité et de surveillance, sont souvent utilisées sans le consentement éclairé des citoyens. Ce phénomène met en lumière le risque d'une surveillance incontournable, où l'individu devient un sujet d’étude, perdant son autonomie.
	Comme cas concret illustrant ce paradoxe est celui de la surveillance gouvernementale dans le cadre de la lutte contre le terrorisme. Les États justifient l'accroissement des mesures de surveillance par la nécessité d'assurer la sécurité publique. Toutefois, cette vigilance constante se fait souvent au détriment de la vie privée des citoyens. Les programmes de surveillance, tels que ceux révélés par Edward Snowden, montrent comment la collecte de données peut être intrusive et comment la transparence peut se transformer en un outil de contrôle. Les citoyens, au lieu de jouir d'une protection renforcée, se retrouvent exposés à des violations de leur intimité, alimentant un climat de méfiance envers les institutions.

	Pour pallier à ce paradoxe, une approche inspirée de l’éthique kantienne peut être envisagée. Kant prône le respect de la dignité humaine et de l'autonomie individuelle. Ainsi, toute action doit être guidée par le principe de la finalité, où chaque individu est traité comme une fin en soi et non comme un simple moyen. Ainsi, dans le contexte de la transparence, cela implique que les données personnelles des citoyens doivent être traitées avec un respect rigoureux. Les gouvernements doivent établir des protocoles transparents et éthiques pour la collecte et l'utilisation des données, garantissant que chaque citoyen soit informé et puisse donner son consentement éclairé.
De plus, il est essentiel de promouvoir une éducation numérique qui sensibilise les citoyens aux enjeux de la transparence et de la vie privée. Ceci en renforçant la capacité des individus à comprendre et à naviguer dans le monde numérique, on les habilite à défendre leurs droits et à participer activement à la construction d'une société où la transparence ne signifie pas l'éradication de l'intimité.
En conclusion, le paradoxe de la transparence soulève des questions fondamentales sur la nature de notre coexistence dans un monde numérique. En conciliant transparence et vie privée à travers une approche éthique, nous pouvons construire une société où la responsabilité et la dignité humaine sont respectées, permettant ainsi à chaque individu de vivre librement et en toute sécurité.

\section{Transformation Ontologique du Numérique}

\subsection*{CHAPITRE 1. Philosophie et Fondements de l’Investigation Numérique}

\begin{itemize}
    \item \textbf{Trace Numérique et Identité.}
    \item \textbf{Manifestation de l'Existence.}
    \item \textbf{Construction de l’Identité Numérique.}
    \item \textbf{Implications Éthiques et Sociales.}
\end{itemize}

\subsection*{Impact sur la Preuve Légale}

\begin{enumerate}
    \item \textbf{Nature des Preuves Numériques.}
    \item \textbf{Authentification et Fiabilité.}
    \item \textbf{Accès et Confidentialité.}
    \item \textbf{Interprétation et Contexte.}
    \item \textbf{Évolution du Droit.}
    \item \textbf{Impact sur la Pratique Judiciaire.}
\end{enumerate}

\section{Mathématiques de l’Investigation}

\subsection{Calcul d’Entropie de Shannon}

Pour calculer l’entropie de plusieurs fichiers, utilisez le script Python suivant :

\begin{verbatim}
import math
def entropy(data):
    prob = [float(data.count(c)) / len(data) for c in set(data)]
    return -sum(p * math.log2(p) for p in prob if p > 0)
\end{verbatim}

\section{Théorie des Graphes en Investigation Criminelle}

\subsection*{Données de Communications Téléphoniques}

\begin{tabular}{|c|c|c|c|}
\hline
Jours & Durée Totale des Appels (minutes) & Appelant & Destinataire \\
\hline
Lundi & 120 & Alice & Bob \\
Mardi & 150 & Charlie & Dave \\
Mercredi & 90 & Eve & Frank \\
Jeudi & 180 & Alice & Charlie \\
Vendredi & 200 & Bob & Eve \\
Samedi & 60 & Frank & Dave \\
Dimanche & 30 & Charlie & Alice \\
\hline
\end{tabular}

% Première figure
\begin{figure}[H]
    \centering
    \includegraphics[width=\textwidth]{visgraphe.PNG} % Remplacez par le nom correct de votre image
    \caption{Graphe des Communications Téléphoniques}
    \label{fig:communications_telephoniques} % Étiquette unique
\end{figure}

% Deuxième figure
\begin{figure}[H]
    \centering
    \includegraphics[width=\textwidth]{calculcentralité.PNG} % Remplacez par le nom correct de votre image
    \caption{Calcul de Centralité}
    \label{fig:calcul_centralite} % Étiquette unique
\end{figure}

\subsection*{Identification de Nœuds Critiques}

\begin{figure}[H]
    \centering
    \includegraphics[width=\textwidth]{codenoeuds.PNG} % Remplacez par le nom correct de votre image
    \caption{Identification des Nœuds Critiques}
    \label{fig:nodes_critics} % Étiquette unique
\end{figure}

\end{document}

Points Clés :

- Étiquettes Uniques : Chaque figure a une étiquette (\label{...}) unique pour éviter les doublons.

- Structure Maintenue : Le contenu et la structure de votre document original ont été préservés.

- Images : N’oubliez pas de vérifier que visgraphe.PNG, calculcentralité.PNG et codenoeuds.PNG sont disponibles dans le répertoire de votre fichier .tex.

Après avoir effectué ces modifications, compilez à nouveau votre document. Cela devrait résoudre le problème de duplication des identifiants et autres erreurs potentielles. Si vous avez d’autres questions, n’hésitez pas à me le faire savoir !